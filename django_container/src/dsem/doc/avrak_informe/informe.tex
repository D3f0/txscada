\documentclass[11pt,a4paper]{report}

\usepackage[utf8]{inputenc}
\usepackage[spanish]{babel}
\usepackage{graphicx}

\begin{document}

\title{Sietema de semaforización Avrak}
\author{Defossé Nahuel, Bruno Zappellini}
\date{\today}

\maketitle

\tableofcontents

\begin{abstract}
El sistema de Semaforización Avrak, basado en el motor de adquisición de
datos\footnote{SCADA} Alsvid, es un software multiplataforma
\footnote{Puede ser ejecutado sobre Windows 2000 y XP, tanto como en Linux}
diseñado con el objeto de monitorear un sistema de semaforización bajo una 
red de microcontroladores Picnet.
Este documento describe como utilizar la GUI\footnote{Interfase Gráfica de
Usuario} así como una breve descripción sobre el significado de cada elemento.
\end{abstract}

\chapter{Introducción}
\label{ch:intro}

Avrak consiste en el Centro de Control. Las funciones que cumple son las
siguientes:
\begin{itemize}
  \item {Creación y edición de Concentradores. Descripto en la
  página \pageref{ch:couc}.}
  \item Cargar archivos SMF\footnote{Archivos de configuración creados con el
  configurador de Ricardo} como Unidades de Control\footnote{La edicón no está
  permitida desde el Centro de Control. Descripto en la página
  \pageref{ch:couc}.}
  
   \begin{item}
   Creación de esquinas sobre un mapa de la ciudad de Trelew con su
   correspondiente configuración:
   consecutivos.
   \begin{itemize}
     \item Selección de concentrador (CO) y unidad de control(UC).
     \item Selección de tipo de esquina y calles.
     \item Inputación de los Movimientos\footnote{Juego de 2 o 3 luces} y
     posicionamiento en el mapa.
   \end{itemize}
   \end{item}
   \item{Visualización en tiempo real de una unidad de ccontrol (UC) para su
  	diagnóstico}
   \item{Listado de eventos. Permitiendo inputar fechas de atención y
   reparación en caso de ser necesario}
   \item 
   \begin{item}
    Filtardo de eventos en función de
    \begin{itemize}
      \item Rango de fechas
      \item Unidade de control
      \item Estado del evento.
    \end{itemize}
   \end{item}
  
\end{itemize}

\chapter{Elementos de la inrerfase}
\label{ch:couc}

\section{Solapa de Configuración}
\subsection{Deslizante de configuración}
\subsubsection{Concentradores}
\begin{center}
	\begin{figure}[h] %[h] para here [b] para bottom [t] para top
	\includegraphics[width=400pt]{./img/co_config.png}
	\end{figure}
\end{center}


\subsubsection{Uniadades de Control}
\section{Deslizante de Comandos}
Envío de comandos

\section{Tabla de eventos}

\newpage
\section{Mapa}
\label{ch:mapa}
El mapa permite visualizar geográficamente la posición del evento.

\begin{center}
	\begin{figure}[h] %[h] para here [b] para bottom [t] para top
	\includegraphics[width=100pt]{./img/middle-mouse-button-press.png}
	\end{figure}
\end{center}



\chapter{Eventos}
\label{ch:eventos}
Un \emph{evento} es un suceso en el campo el cual requiere atención por el
usuario. Algunos de los sucesos que generan eventos son lámparas quemadas,
apertura de llaves por bandalismo, entre otros.

Los eventos se encuentran divididos en 2 tipos:
\begin{itemize}
  \item Alta prioridad
  \item Baja prioridad
  
\end{itemize}




Los eventos tienen las siguientes propiedades:
\begin{description}
  \item [COID] { Identificador de concentrador y unidad de control }
  \item [Tipo] { Tipo de evento }
  \item [Prioridad] { Prioridad }
  \item [Nombre] { Nombre de la unidad de control que reportó el evento}
  \item [Descripción] { En función a campos internos del evento, se muestra una
  descripción del evento }
  \item [Puerto/Movi] { Número de movimiento o número de puerto }
  \item [Lámpara/Bit] { Número de lampara en el movimiento o número de bit del
  puerto que produjo el suceso }
  \item [Estado] { Los eventos son producidos por cambios de estado lógico, el
  nuevo estado es el que se inidica en este campo }
  \item [Timestamp] { Fecha y hora en la cual fue reportado el evento con
  precisicón de  décima de milisegundo }
  \item [Atención] { Hora de atención, el operario, \ldots}
  \item [Reparación] { Al realizarse la reparación del evento en el campo, el
  operario mediante le botón derecho del mouse puede establecer el evento como
  reparado }
  
\end{description}


\chapter{Reportes}
\label{ch:print}
Capítulo sobre la salida impresa.


\end{document}